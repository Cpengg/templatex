%=================================================================
% gitlatexdiff
%
%  https://gitlab.com/git-latexdiff/git-latexdiff
%=================================================================
%  git latexdiff HEAD  HEAD~5 --main templatex.tex
%  git latexdiff HEAD~1  --main templatex.tex
%  View pdf to see difference
%
%=================================================================
%
% Todo Notes for marginal comments
% 
%\newcount\DraftStatus  % 0 suppresses notes to selves in text
%\DraftStatus=1   % TODO: set to 0 for final version
\ifnum\DraftStatus=1
	\usepackage[draft,colorinlistoftodos,color=orange!30]{todonotes}
\else
	\usepackage[disable,colorinlistoftodos,color=blue!30]{todonotes}
\fi 
%\usepackage[disable]{todonotes} % notes not showed
%\usepackage[draft]{todonotes}   % notes showed
%
\makeatletter
 \providecommand\@dotsep{5}
 \def\listtodoname{List of Todos}
 \def\listoftodos{\@starttoc{tdo}\listtodoname}
 \makeatother
%
%=================================================================
%
\usepackage{color}
\newcommand{\draftnote}[3]{ 
	\todo[author=#2,color=#1!30,size=\footnotesize]{\textsf{#3}}	}
% TODO: add yourself here:
%
\newcommand{\gangli}[1]{\draftnote{blue}{GLi:}{#1}}
\newcommand{\qwu}[1]{\draftnote{red}{QWu:}{#1}}
\newcommand{\gliMarker}
	{\todo[author=GLi,size=\tiny,inline,color=blue!40]
	{Gang Li has worked up to here.}}
\newcommand{\qwuMarker}
	{\todo[author=Cpengg,size=\tiny,inline,color=blue!40]
	{Qiong Wu has worked up to here.}}
%=================================================================

%=================================================================
%
% general packages
%  https://en.wikibooks.org/wiki/Category:Book:LaTeX
%  https://en.wikibooks.org/wiki/LaTeX/Package_Reference
%
%=================================================================
\usepackage{graphicx}
\usepackage{algorithm}
\usepackage{algorithmic}
\usepackage{breqn}
\usepackage{subcaption}
\usepackage{multirow}
\usepackage{psfrag}
\usepackage{url}
\usepackage{hyperref}
%\usepackage[colorlinks]{hyperref}
%\usepackage{cite}
\usepackage{cleveref}
\usepackage{booktabs}
\usepackage{rotating}
\usepackage{colortbl}
\usepackage{paralist}
%\usepackage{geometry}
\usepackage{epstopdf}
\usepackage{nag}
\usepackage{microtype}
\usepackage{siunitx}
\usepackage{nicefrac}
\usepackage{breakurl}
\usepackage{fontawesome}
\usepackage{xcolor}
\usepackage{multicol}
\usepackage{wrapfig}
\usepackage{todonotes}
\usepackage{tablefootnote}
\usepackage{threeparttable}
% for random text
\usepackage{lipsum}
\usepackage[english]{babel}
\usepackage[pangram]{blindtext}
% for tikz figures
\usepackage{tikz}
\usetikzlibrary{fit,positioning,arrows.meta,shapes,arrows}
%\tikzset{neuron/.style={circle,thick,fill=black!25,minimum size=17pt,inner sep=0pt},
%	input neuron/.style={neuron, draw,thick, fill=gray!30},
%	hidden neuron/.style={neuron,fill=white,draw},
%	hoz/.style={rotate=-90}}
%
%=================================================================



\begin{TulipStyle}
\usepackage[numbers]{natbib}
%=================================================================
%
% Version control information
%
%=================================================================
\usepackage{gitinfo2}
%=================================================================
\usepackage{fancyhdr}
\pagestyle{fancy}
\fancyhead{} % clear all header fields
\fancyhead[RO,LE]{\textsl{\rightmark}}
\fancyhead[LO,RE]{\ensuremath{\Rightarrow}
		\textbf{\textbf{[CONFIDENTIAL]}}\ensuremath{\Leftarrow}}
\fancyhead[CO,CE]{}
%=================================================================
\fancyfoot{} % clear all footer fields
\fancyfoot[CE,CO]{\textbf{\thepage}} 
\fancyfoot[LO,LE]{\includegraphics[height=.9\headheight]{logos/tulip-logo.eps}
		\gitVtagn-\gitBranch\ (\gitCommitterDate)}
\fancyfoot[RO,RE]{Committed by: \textsl{\gitCommitterName}}

\setlength{\headheight}{12pt}
\renewcommand{\headrulewidth}{0.4pt}
\renewcommand{\footrulewidth}{0.4pt}
%=================================================================


%=================================================================
% for math notations
% ----------------------------------------------------------------
\usepackage{mathtools}
\usepackage{amsthm}
%
% THEOREMS -------------------------------------------------------
%
\newtheorem{thm}{Theorem}[section]
\newtheorem{cor}[thm]{Corollary}
\newtheorem{lem}[thm]{Lemma}
\newtheorem{prop}[thm]{Proposition}
\theoremstyle{definition}
\newtheorem{defn}[thm]{Definition}
\theoremstyle{remark}
\newtheorem{rem}[thm]{Remark}
\numberwithin{equation}{section}
% MATH -----------------------------------------------------------
\newcommand{\norm}[1]{\left\Vert#1\right\Vert}
\newcommand{\abs}[1]{\left\vert#1\right\vert}
\newcommand{\set}[1]{\left\{#1\right\}}
\newcommand{\Real}{\mathbb R}
\newcommand{\eps}{\varepsilon}
\newcommand{\To}{\longrightarrow}
\newcommand{\BX}{\mathbf{B}(X)}
% ----------------------------------------------------------------
\newcommand{\I}{{\cal I}}
\newcommand{\Id}{{\cal I} }
\newcommand{\Dc}{{\cal D}}
\newcommand{\J}{{\cal J}}
\newcommand{\Dn}{{\cal D}_n}
\newcommand{\Dd}{{\cal D}_n }
\renewcommand{\P}{{\cal P}}
\newcommand{\Nu}{{\cal N} }
\newcommand{\B}{{\cal B}}
\newcommand{\Bf}{{\bf B}}
\newcommand{\Y}{{\bf Y}}
\newcommand{\A}{{\cal A}}
% ----------------------------------------------------------------
\newcommand{\V}{{\cal V}}
\newcommand{\M}{{\cal M}}
\newcommand{\F}{{\cal F}}
\newcommand{\Fd}{{\cal F}}
\newcommand{\BF}{{\cal BF}_n}
\newcommand{\BFd}{{\cal BF}_n}
\newcommand{\TF}{{\cal TF}_n}
\newcommand{\TFd}{{\cal TF}_n}
%\newcommand{\G}{{\cal G}}
\newcommand{\X}{{\cal X}}
\newcommand{\E}{{\cal E}}
\newcommand{\K}{{\cal K}}
\newcommand{\T}{{\cal T}_n}
\renewcommand{\H}{{\cal H}}
% ----------------------------------------------------------------
\newtheorem{Remark}{Remark}
\newtheorem{proposition}{Proposition}
\newtheorem{theorem}{Theorem}
\newtheorem{lemma}{Lemma}
\newtheorem{corollary}{Corollary}
\newtheorem{example}{Example}
\newtheorem{definition}{Definition}
\newtheorem{Algorithms}{Algorithm}
% ----------------------------------------------------------------
\newcommand{\bu}{{\mathbf 1} }
\newcommand{\bo}{{\mathbf 0} }
\newcommand{\N}{\mbox{{\sl l}}\!\mbox{{\sl N}}}
% ----------------------------------------------------------------
\def\uint{[0,1]}
\def\proof{{\scshape Proof}. \ignorespaces}
\def\endproof{{\hfill \vbox{\hrule\hbox{%
   \vrule height1.3ex\hskip1.0ex\vrule}\hrule
  }}\par}
%
%=================================================================

\hypersetup
{
    pdfauthor={\gitAuthorName},
    pdfsubject={TULIP Lab},
    pdftitle={},
    pdfkeywords={TULIP Lab, Data Science},
%	bookmarks=true,  
}

\end{TulipStyle}

